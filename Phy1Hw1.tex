\documentclass{article}

\usepackage{amsmath}
\usepackage{amssymb}
\usepackage{textcmds}
\usepackage{graphicx}
\usepackage{subcaption}
\usepackage{float}
\usepackage{hhline}
\usepackage[parfill]{parskip}

\title{PHY201: Homework 1}

\date{7$^{th}$ November 2019}
\author{Ariel Attias\\Matthieu Chapuy\\Matthieu Melennec\\Andr\'e Renom}

\begin{document}


	\pagenumbering{gobble}
	\maketitle
	\tableofcontents
	\newpage
	\pagenumbering{arabic}

\section{Horizontally Excited Pendulum}

\subsection{} % Exercise 1.1

Under normal assumptions, we would write the position of the pendulum as:
\begin{align*}
	\left( \begin{matrix} x \\ z \end{matrix} \right) = \left( \begin{matrix}x_f +  l\, sin\,\theta \\ -l\, cos\,\theta \end{matrix} \right)
\end{align*}
However for $\theta \ll 1$, we can use the Taylor series of the trigonometric functions to approximate $cos\theta \approx 1$ and $sin\theta \approx \theta$. We therefore have:
\begin{align*}
	\left( \begin{matrix} x \\ z \end{matrix} \right) = \left( \begin{matrix}x_f +  l\theta \\ -l \end{matrix} \right)
\end{align*}

\subsection{} % Exercise 1.2

We now consider the forces in the direction of the bar at the point mass. We observe two forces: the constraint force in the bar, that acts parallel to the bar, and the force due to gravity, that acts straight down. By projecting the force due to gravity along the direction of the bar, we can write the equation
\begin{align*}
	F_{bar} - mg\,cos\,\theta = ma
\end{align*}
with $a$ the component of the acceleration in the direction of the bar. However we know the bar to be rigid, hence $a = 0$, and as established before, over the studied range, $cos\theta = 1$. We can therefore re-write the equation above as
\begin{align*}
	F_{bar} = mg
\end{align*}

\subsection{} % Exercise 1.3

In order to write the oscillator equation of the system, we will consider the motion of the mass in the direction of the $x$-axis. From our definition of the $x$ position of the particle in \textbf{1.1}, we can derive twice to obtain the acceleration of the particle in the $x$-direction, $\ddot{x} = \ddot{x}_f + l \ddot{\theta}$.\\

\noindent By considering the forces on the mass, we get that:
\begin{align*}
	-F_{bar}sin\theta - \eta \dot{x} &= ma\\
	-mg\,sin\theta - \eta \dot{x} &= m( \ddot{x}_f + l \ddot{\theta})\\
	-g\,sin\theta - \frac{\eta}{m} (\dot{x}_f + l\dot{\theta}) &= -\omega^2x_0sin \, \omega t + l \ddot{\theta}
\end{align*}
From this we can write our oscillator equation:
\begin{align*}
	 \frac{d^2 \theta}{dt^2} + \frac{\eta}{m}\frac{d \theta}{dt} + \frac{g}{l}\theta  = \frac{\omega^2}{l}x_0sin\,\omega t - \frac{\eta}{m}x_0cos\,\omega t 
\end{align*}

\subsection{} % Exercise 1.4
For simplicity, we are going to replace $\frac{\eta}{m}$ by $\gamma$ and $\frac{g}{l}$ by $\omega_0$
This gives the following expression :
$$\ddot{\theta} + \gamma\dot{\theta} + \omega_0^2\theta = \frac{\gamma}{l}\omega x_0 cos(\omega t) + \frac{\omega^2}{l}x_0sin(\omega t) $$
Replacing the expression of $x_f(t)$ by its complex equivalent and searching solutions for $\theta = \Theta e^{i\gamma t}$ we get the equation :

\begin{align*}
-\omega^2\Theta e^{i\omega t} + i\gamma \omega \Theta e^{i\omega t} + \omega_0^2\Theta e^{i\omega t} &= \frac{\gamma}{l}\omega x_0 cos(\omega t) + \frac{\omega^2}{l}x_0sin(\omega t) \\
\Leftrightarrow \Theta ((\omega_0^2 - \omega ^2) + i\gamma \omega) &= \frac{\gamma}{l} \omega x_0 + \frac{\omega^2}{l} ix_0 \\
\Leftrightarrow \Theta &= \frac{\gamma \omega x_0 + \omega^2 i x_0}{l[(\omega_0^2 - \omega^2) + i\gamma \omega]} \\
&=\frac{(\gamma \omega x_0 + \omega^2 i x_0)((\omega_0^2-\omega^2)-i\gamma \omega)}{l(\omega_0^2 - \omega^2)^2 + \gamma^2 \omega^2l} \\
&= \frac{\gamma \omega x_0 (\omega_0^2 - \omega^2) + \omega^3 x_0 \gamma) + i(\omega^2x_0(\omega_0^2-\omega^2) - \gamma^2\omega^2x_0)}{l(\omega_0^2 - \omega^2)^2 + \gamma^2 \omega^2l} \\
\end{align*}

We are going to let this expression as it is for question 1.4, as we it will be simplified in question 1.5

\subsection{} %Exercise 1.5

We now write the magnitude of the transfert function of the system $H(\omega) = \left|\frac{\theta}{x_f}\right|$

We have : 

\begin{align*}
H(\omega) &= \frac{\theta}{x_p}\\
&= -i\frac{\Theta}{x_0}\\
&= \frac{\omega^2-(\omega_0^2 -\omega^2) - \gamma^2\omega^2) - i(\gamma \omega(\omega_0^2-\omega^2)+ \omega^3\gamma)}{l(\omega_0^2 - \omega^2)^2 + \gamma^2\omega^2l} 
\end{align*}

Taking the modulos it yields : 
\begin{align*}
\left|H(\omega)\right| &= \frac{1}{l(\omega_0^2-\omega^2)^2+\gamma^2\omega^2l}\left[(\omega^2(\omega_0^2-\omega^2)-\gamma^2\omega^2)^2 + (\gamma\omega(\omega_0^2-\omega^2+\omega^2\gamma)^2\right]^{\frac{1}{2}} \\
&= \frac{1}{l(\omega_0^2-\omega^2)^2+\gamma^2\omega^2l}\left[\omega^4(\omega_0^2-\omega^2)^2+\gamma^4\omega^4+\gamma^2\omega^2(\omega_0^2-\omega^2)^2+\omega^6\gamma^2\right]^{\frac{1}{2}} \\
\end{align*}

In order to study the sign of this expression, we have going to derivate $\left|H(\omega)\right|$ with respect to $\gamma$, this yields :

\begin{align*}
\frac{d\left|H(\omega)\right|}{d\gamma} &= -\frac{2\gamma \omega^2}{(\omega_0^2-\omega^2)^2 + \gamma^2\omega^2}\left[\omega^4(\omega_0^2-\omega^2)^2+\gamma^4\omega^4+\gamma^2\omega^2(\omega_0^2-\omega^2)^2+\omega^6\gamma^2\right]^{\frac{1}{2}}\\ &+ \frac{1}{l(\omega_0^2-\omega^2)^2+\gamma^2\omega^2l}\frac{2\gamma^3\omega^4+\gamma \omega^2(\omega_0^2-\omega^2)^2+\gamma \omega^6}{\left[\omega^4(\omega_0^2-\omega^2)^2+\gamma^4\omega^4+\gamma^2\omega^2(\omega_0^2-\omega^2)^2+\omega^6\gamma^2\right]^{\frac{1}{2}}}
\end{align*}


But, we know that :
\begin{align*}
 & -\frac{2\gamma \omega^2}{(\omega_0^2-\omega^2)^2 + \gamma^2\omega^2}\left[\omega^4(\omega_0^2-\omega^2)^2+\gamma^4\omega^4+\gamma^2\omega^2(\omega_0^2-\omega^2)^2+\omega^6\gamma^2\right]^{\frac{1}{2}}\\
&\ge  \frac{1}{l(\omega_0^2-\omega^2)^2+\gamma^2\omega^2l}\frac{2\gamma^3\omega^4+\gamma \omega^2(\omega_0^2-\omega^2)^2+\gamma \omega^6}{\left[\omega^4(\omega_0^2-\omega^2)^2+\gamma^4\omega^4+\gamma^2\omega^2(\omega_0^2-\omega^2)^2+\omega^6\gamma^2\right]^{\frac{1}{2}}}
\end{align*}

Hence we have that $\frac{d\left|H(\omega)\right|}{d\gamma} \le 0$.
Thus as $\eta$ increases, it means that $\left|H(\omega)\right|$ decreases, which confirms our intuition that the amplitude should increases when the friction coefficent is small.

\subsection{} %Exercise 1.6

Taking the limit of the expression we found above at $\eta$ tends to 0 yields :
$$\lim_{\eta \to 0}\left|H(\omega)\right| = \frac{\omega}{l(\omega_0^2-\omega^2)} $$

Hence this system is not conservative JE VOUS LAISE EXPLIQUER POURQUOI


\section{Equilibrium of Two Masses}

\subsection{} % Exercise 2.1

In order to use $z_1$ as a generalised coordinate for the system, we would have to be able to derive$u_2$ from it. However, since the two masses are connected by an inextensible filament of length $l$ and placed on a rigid surface, this is in fact the case. Specifically:
\begin{align*}
	u_2 = l - z_1
\end{align*}

\subsection{} % Exercise 2.2

We will consider the 0 gravitational potential point to be at the level of the pulley.
\begin{align*}y
	\mathcal{L} &= E_{K} - E_P\\
	\mathcal{L}(z,\dot{z},t) &= \frac{1}{2}(m_1+m_2)\dot{z}^2 - g(m_1z + m_2(l-z)sin\,\alpha)\\
	&= \frac{1}{2}(m_1+m_2)\dot{z}^2 - (m_2sin\,\alpha - m_1)g z - m_2 g l sin(\alpha)
\end{align*}

\subsection{} % Exercise 2.3

We consider the Euler-Lagrange equation:
\begin{align*}
	\frac{\partial \mathcal{L}}{\partial z} &= \frac{d}{dt}\frac{\partial \mathcal{L}}{\partial \dot{z}}\\
	 (m_2sin\,\alpha - m_1)g &= \frac{d}{dt}(m_1 + m_2)\dot{z}\\
	 (m_2sin\,\alpha - m_1)g &= (m_1 + m_2)\ddot{z}
\end{align*}

\subsection{} % Exercise 2.4

From the Euler-Lagrange equation gives us that:
\begin{align*}
	\ddot{z} = \frac{ m_2gsin\,\alpha  - m_1}{m_1 + m_2}
\end{align*}
In order to have equilibrium, we need $\ddot{z} = 0$. This can only be obtained if $m_2sin\,\alpha = m_1$. Since $sin\,\alpha \leq 1$, This is possible if $m_2 > m_1$ but impossible if $m_2 < m_1$.

\section{Suspended Bar}

\subsection{One Mass Only}

\subsubsection{} % Exercise 3.1.1

In order to define the gravitaitonal potential $V_g$, we will first define the 0 potential point to be when the pendulum is at rest at $r_0$. We therefore want to find the difference in potential between a given point $(r,\theta)$ and $(r_0,0)$. In order to do this, we project r along the vertical, and take the difference with $r_0$. This yields:
\begin{align*}
	V_g &= -mgh\\
	&= -mg(r\,cos\,\theta - r_0)
\end{align*}

\subsubsection{} % Exercise 3.1.2

We know that the total potential is the sum of the gravitational and elastic potential, we write this as:
\begin{align*}
	V &= V_g + V_e\\
	&=  -mg(r\,cos\,\theta - r_0) + V_e
\end{align*}
In order to find the elastic potential, we need the extension of the spring. This is simply $r - r_0$. Therefore
\begin{align*}
	V &=  -mg(r\,cos\,\theta - r_0) + \frac{1}{2}k(r-r_0)^2
\end{align*}

\subsubsection{} % Exercise 3.1.3

In order to write the kinetic energy, we need to combine the kinetic energy in $\vec{e_r}$ and $\vec{e}_{\theta}$. We therefore have:
\begin{align*}
	E_K &= \frac{1}{2}m\dot{r}^2 + \frac{1}{2}m(r\dot{\theta})^2\\
	&= \frac{1}{2}m(\dot{r}^2 + r^2\dot{\theta}^2)
\end{align*}

\subsubsection{} % Exercise 3.1.4

\begin{align*}
	\mathcal{L} &= E_{K} - E_P\\
	\mathcal{L}(r,\theta,\dot{r},\dot{\theta},t) &= \frac{1}{2}m(\dot{r}^2 + r^2\dot{\theta}^2) + mg(r\,cos\,\theta - r_0) - \frac{1}{2}k(r-r_0)^2
\end{align*}

\subsubsection{} % Exercise 3.1.5

Defining $\tilde{r} = r - r_0 \ll 1$, we have $\dot{\tilde{r}} = \dot{r}$. We will first show that we can say that $r^2 \approx r_0^2$
\begin{align*}
	r^2 &= r_0^2 + 2\tilde{r}r_0 + \tilde{r}^2\\
	&= r_0^2\left( 1 + 2\frac{\tilde{r}}{r} +\frac{\tilde{r}^2}{r_0^2} \right)
\end{align*}
Since $\tilde{r} \ll 1$, we can neglect the second and third term, leaving us with $r^2 \approx r_0^2$.\\
With $\theta \ll 1$, we have that $cos\,\theta = 1 - \frac{\theta}{2}$. We therefore re-write the equation above with these considerations.
\begin{align*}
	L &\approx \frac{1}{2}m(\dot{\tilde{r}}^2 + r_0^2\dot{\theta}^2) + mg(r(1- \frac{\theta^2}{2})- r_0) - \frac{1}{2}k\tilde{r}^2\\
	&\approx \frac{1}{2}m(\dot{\tilde{r}}^2 + r_0^2\dot{\theta}^2)  - \frac{1}{2}k\tilde{r}^2 +mg\tilde{r} - \frac{1}{2}mgr_0\theta^2
\end{align*}

\subsubsection{} % Exercise 3.1.6

We write the Euler-Lagrange for $\tilde{r}$.
\begin{align*}
	\frac{\partial L}{\partial \tilde{r}} &= \frac{d}{dt}\frac{\partial L}{\partial \dot{\tilde{r}}}\\
	mg - k\tilde{r} &= m\ddot{\tilde{r}}
\end{align*}
We now do the same for $\theta$.
\begin{align*}
	\frac{\partial L}{\partial \theta} &= \frac{d}{dt}\frac{\partial L}{\partial \dot{\theta}}\\
	-mgr_0\theta &= mr_0^2\ddot{\theta}
\end{align*}
\subsubsection{} % Exercise 3.1.7

We write the equations of motion for $\tilde{r}$ and $\theta$.
\begin{align*}
	\ddot{\tilde{r}} &= g - \frac{k}{m}\tilde{r}\\
	\ddot{\theta} &= -\frac{g}{r_0}\theta
\end{align*}
We recognise these equations as those of a harmonic oscillator. We therefore have two modes, one for the spring oscillator, and one for the pendulum. The frequency of the spring node is $\frac{1}{2\pi}\sqrt{\frac{g}{r_0}}$, and for the pendulum it is $\frac{1}{2\pi}\sqrt{\frac{k}{m}}$.

\subsection{Connected Masses}

\subsubsection{} % Exercise 3.2.1

The system has three degrees of freedom, vertical, horizontal, and rotational in the plane. It has one holonomic constraint, specifically the distance between the two masses is constant, as it is secured by a rigid massless rod.

\subsubsection{} % Exercise 3.2.2

We consider the holonomic constraint by separating the horizontal and vertical components of the distance between the two masses:
\begin{align*}
	L_H^2 + L_V^2 &= (r_2\,cos\,\theta_2 - r_1 \, cos \,\theta_1)^2 + (r_2\,sin\,\theta_2 - r_1\,sin\,\theta_1 + L)^2\\
	&= r_2^2 (cos^2\theta_2 + sin^2 \theta_2) + r_1^2(cos^2\theta_1 + sin^2 \theta_1) +L^2 + 2L(r_2\,sin\,\theta_2 - r_1\,sin\,\theta_1)\\ & \quad - 2r_1r_2(cos\,\theta_2\,cos\,\theta_1 + sin\,\theta_2\,sin\,\theta_1)
\end{align*}
By using trigonometric identities, we can re-write this as:
\begin{align*}
	L_H^2 + L_V^2 &=r_2^2+r_1^2 - 2r_1r_2cos(\theta_1 - \theta_2) + 2L(r_2\,sin\,\theta_2 - r_1\,sin\,\theta_1) + L^2
\end{align*}
We then simp[lify using the small angle approximations.
\begin{align*}
	L_H^2 + L_V^2 &= (\tilde{r}_1 -\tilde{r}_2)^2 - 2(r_0+\tilde{r}_1)(r_0 + \tilde{r}_2)(1 - 1 - \frac{(\theta_1 - \theta_2)^2}{2})\\ &\quad + 2L\big((r_0 + \tilde{r}_2)\theta_2 - (r_0 + \tilde{r}_1)\theta_1 \big) + L^2\\
 &= (\tilde{r}_1 -\tilde{r}_2)^2 + 2Lr_0(\theta_2 - \theta_1) + (r_0 + \tilde{r}_1)(r_0 + \tilde{r}_2)(\theta_2^2 -2\theta_1\theta_2 + \theta_1^2) +L^2\\
&=  (\tilde{r}_1 -\tilde{r}_2)^2 + 2Lr_0(\theta_2 - \theta_1) + r_0^2(\theta_2 - \theta_1)^2 + L^2
\end{align*}
But we know that $L_H^2 + L_V^2 = L^2$, therefore:
\begin{align*}
	(\tilde{r}_1 -\tilde{r}_2)^2 + 2Lr_0(\theta_2 - \theta_1) + r_0^2(\theta_2 - \theta_1)^2 = 0
\end{align*}

\subsubsection{} % Exercise 3.2.3

We first consider the potential energies:
\begin{align*}
	V_{tot} = \frac{1}{2}k\tilde{r}_1^2 + \frac{1}{2}k\tilde{r}_2^2 + \frac{mgr_0}{2}(\theta_1^2 + \theta^2_2) - mg(\tilde{r}_1 + \tilde{r}_2)
\end{align*}
We now consider the kinetic energy:
\begin{align*}
	T = \frac{1}{2}m(\dot{\tilde{r}}_1^2 + \dot{\tilde{r}}_2^2 + r_0^2(\dot{\theta}_1^2 + \dot{\theta}_2^2))
\end{align*}
We therefore write the Lagrangian as:
\begin{align*}
	\mathcal{L} = \frac{1}{2}m(\dot{\tilde{r}}_1^2 + \dot{\tilde{r}}_2^2 + r_0^2(\dot{\theta}_1^2 + \dot{\theta}_2^2)) - \frac{1}{2}k\tilde{r}_1^2 + \frac{1}{2}k\tilde{r}_2^2 + \frac{mgr_0}{2}(\theta_1^2 + \theta^2_2) - mg(\tilde{r}_1 + \tilde{r}_2)
\end{align*}

We know write the Euler Lagrange equations with $f =  (\tilde{r}_1 -\tilde{r}_2)^2 + 2Lr_0(\theta_2 - \theta_1) + r_0^2(\theta_2 - \theta_1)^2 = 0$.
\begin{align*}
	\frac{\partial L}{\partial \tilde{r}_1} + \lambda\frac{\partial f}{\partial \tilde{r}_1} &= \frac{d}{dt}\frac{\partial L}{\partial \dot{\tilde{r}}_1}\\
	-k\tilde{r}_1 + mg + 2\lambda(\tilde{r}_1 - \tilde{r}_2) = m\ddot{\tilde{r}}_1
\end{align*}
similarly, we get:
\begin{align*}
	-k\tilde{r}_2 + mg + 2\lambda(\tilde{r}_2 - \tilde{r}_1) = m\ddot{\tilde{r}}_2
\end{align*}
Using the change of variable $R = \tilde{r}_1 + \tilde{r}_2$, we sum the two equations to get:
\begin{align*}
	m\ddot{R} + kR = 2mg
\end{align*}
We now consider the same for $\theta$:
\begin{align*}
	\frac{\partial L}{\partial \theta_1} + \lambda\frac{\partial f}{\partial \theta_1} &= \frac{d}{dt}\frac{\partial L}{\partial \dot{\theta}_1}\\
	-mgr_0\theta_1 + \lambda(2r_0^2(\theta_1 - \theta_2) - 2Lr_0) &= mr_0^2\ddot{\theta}_1
\end{align*}
Similarly
\begin{align*}
	-mgr_0\theta_2 + \lambda(2r_0^2(\theta_2 - \theta_1) - 2Lr_0) = mr_0^2\ddot{\theta}_2
\end{align*}
Using the change of variable $\Theta = \theta_1 + \theta_2$, we sum the two equations to get:
\begin{align*}
	mr_0^2\ddot{\Theta} = -mgr_0\Theta
\end{align*}
\subsubsection{} % Exercise 3.2.4

Under first order approximations, the holonomic constraint is:
\begin{align*}
	2Lr_0(\theta_2 - \theta_1) = 0
\end{align*}
Therefore it is clear that we have $\theta_1 = \theta_2$. Reinjecting this into the second order approximation of the constraint, we also get that $\tilde{r}_1 = \tilde{r}_2$. However for now, we have not been able to prove the third mode. Physically, we see that these three mode correspond to:


\end{document}
